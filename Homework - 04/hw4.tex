\documentclass[12pt]{article}
\usepackage{fullpage}
\usepackage[top=10mm, bottom=20mm, left=10mm, right=10mm]{geometry}
\usepackage{amsmath,amsthm,amssymb}
\usepackage{lastpage}
\usepackage{enumerate}
\usepackage{fancyhdr}
\usepackage{xcolor}
\usepackage{graphicx}
\usepackage{listings}
\usepackage{hyperref}
\usepackage{answers}
\usepackage{setspace}
\usepackage{enumitem}
\usepackage{multicol}
\usepackage{mathrsfs}
\usepackage{algorithmic}
\usepackage{stmaryrd}
\usepackage[ruled,linesnumbered,vlined]{algorithm2e}
\usepackage{tikz}
\usetikzlibrary{automata, positioning}

\hypersetup{%
  colorlinks=true,
  linkcolor=blue,
  linkbordercolor={0 0 1}
}

\lstdefinestyle{Python}{
    language        = Python,
    frame           = lines, 
    basicstyle      = \footnotesize,
    keywordstyle    = \color{blue},
    stringstyle     = \color{green},
    commentstyle    = \color{red}\ttfamily
}

\setlength{\parindent}{0.0in}
\setlength{\parskip}{0.05in}

\newcommand\course{\textbf{EE 456}}   
\newcommand\name{Aishwarye Omer}     

\pagestyle{fancyplain}
\headheight 35pt
\lhead{\name\\\course{}}
\chead{\textbf{\Large Homework - 04}}
\rhead{\today}
\lfoot{}
\cfoot{}
\rfoot{\small\thepage}
\headsep 1.5em

\newlength\myindent
\setlength\myindent{2em}
\newcommand\bindent{%
  \begingroup
  \setlength{\itemindent}{\myindent}
  \addtolength{\algorithmicindent}{\myindent}
}
\newcommand\eindent{\endgroup}

\newenvironment{solution}[1][Solution]{\begin{trivlist}
\item[\hskip \labelsep {\bfseries #1}]}{\end{trivlist}}


\begin{document}

\textbf{Question: } \text{Calculate the energy value of the Bi-directional Hetero-Associator as given in the lecture}
\BlankLine	
\textbf{Solution :} The energy function is given as 
\begin{centering}
	$E$ = $-0.5 \times \displaystyle \sum_{i,j} X_i \times W_{ij} \times Y_j$ $-$ $\displaystyle \sum_{i} X_i \times Y_i + \sum_{i} \theta_i \times Y_i$ 
\end{centering}. 
\BlankLine
The third term in the above formula for BAM network is 0  i.e. $\sum_{i} \theta_i \times Y_i = 0$
\BlankLine
For a BAM network the appropriate energy function is the average of the signal energy for a forward and backward pass which is $E$ = $-0.5 \times \displaystyle \sum_{i,j}  X_i \times W_{ij} \times Y_j$ $-$ $\displaystyle \sum_{i} X_i \times Y_i$
\begin{enumerate}[label=\textbf{\alph* .}]
	
	\item Following are the inputs and synaptic weight matrix:
	\BlankLine
	$Y_1 = $
	$\begin{bmatrix}
		-1\quad  \\
		\phantom{-}1\quad \\
	\end{bmatrix}$ , $Y_2 = $
	$\begin{bmatrix}
		\quad1\quad  \\
		\quad1\quad \\
	\end{bmatrix}$ , $W = $ 
	$\begin{bmatrix}
		\phantom{-}0 & -2\quad\\
		\phantom{-}0 & \phantom{-}2\quad\\
		\phantom{-}2 & \phantom{-}0\quad\\
		\phantom{-}0 & \phantom{-}2\quad\\
		\phantom{-}0 & -2\quad\\
		-2 & \phantom{-}0\quad\\
		\phantom{-}0 & \phantom{-}2\quad\\
		-2 & \phantom{-}0\quad\\
		-2 & \phantom{-}0\quad\\
		\phantom{-}0 & \phantom{-}2\quad\\
		\phantom{-}0 & -2\quad\\
		-2 & \phantom{-}0\quad\\
		-2 & \phantom{-}0\quad\\
		\phantom{-}2 & \phantom{-}0\quad\\
		\phantom{-}0 & \phantom{-}2\quad\\
	\end{bmatrix}$, $X_1 = $
		$\begin{bmatrix}
		-1 \quad\\
		\phantom{-}1 \quad\\
		-1 \quad\\
		\phantom{-}1 \quad\\
		-1 \quad\\
		\phantom{-}1 \quad\\
		\phantom{-}1 \quad\\
		\phantom{-}1 \quad\\
		\phantom{-}1 \quad\\
		\phantom{-}1 \quad\\
		-1 \quad\\
		\phantom{-}1 \quad\\
		\phantom{-}1 \quad\\
		-1 \quad\\
		\phantom{-}1 \quad\\
	\end{bmatrix}$ and $X_2 = $
	$\begin{bmatrix}
		-1 \quad\\
		\phantom{-}1 \quad\\
		\phantom{-}1 \quad\\
		\phantom{-}1 \quad\\
		-1 \quad\\
		-1 \quad\\
		\phantom{-}1 \quad\\
		-1 \quad\\
		-1 \quad\\
		\phantom{-}1 \quad\\
		-1 \quad\\
		-1 \quad\\
		-1 \quad\\
		\phantom{-}1 \quad\\
		\phantom{-}1 \quad\\
	\end{bmatrix}$

	
	\item The first term is $-0.5 \times \displaystyle \sum_{i,j}^{j \neq i} X_i \times W_{ij} \times Y_j$ . According to the above values
		\begin{itemize}[label=$\bullet$]
			\item $ -0.5 \times [\quad ( X_1 \times W \times Y_1 ) + ( X_2 \times W \times Y_2) +  ( X_1 \times W \times Y_2 ) + ( X_2 \times W \times Y_1 )\quad]$
			\item $ = -0.5 \times (16 + 14 + 16 + 14)$
			\item $ = -30$
		\end{itemize}
	
	\item The second term is $( \displaystyle \sum_{i} X_i \times Y_i)$
		\begin{itemize}[label=$\bullet$]
			\item $ [\quad (X_1 \times Y_1) + (X_2 \times Y_2)  + (X_1 \times Y_2) + (X_2 \times Y_1) \quad]$ \quad( multiplication element wise )
			\item $ = (-5 + 5 - 1 - 1) $
			\item $ = -2$
		\end{itemize}
	
	\item Therefore energy function for the above BAM network evaluates to 
		\begin{itemize}[label=$\bullet$]
			\item $ E = -30 -(-2) + 0$
			\item $ E = -30 + 2 $
			\item $ E = -28 $
		\end{itemize}

	
\end{enumerate}



\end{document}