
% This LaTeX was auto-generated from MATLAB code.
% To make changes, update the MATLAB code and republish this document.
\documentclass[12pt]{article}
\usepackage{fullpage}
\usepackage[top=10mm, bottom=37mm, left=10mm, right=10mm]{geometry}
\usepackage{amsmath,amsthm,amssymb}
\usepackage{lastpage}
\usepackage{enumerate}
\usepackage{fancyhdr}
\usepackage{xcolor}
\usepackage{graphicx}
\usepackage{listings}
\usepackage{hyperref}
\usepackage{answers}
\usepackage{setspace}
\usepackage{enumitem}
\usepackage{multicol}
\usepackage{mathrsfs}
\usepackage{algorithmic}
\usepackage{stmaryrd}
\usepackage[ruled,linesnumbered,vlined]{algorithm2e}
\usepackage{tikz}
\usetikzlibrary{automata, positioning}

\hypersetup{%
	colorlinks=true,
	linkcolor=blue,
	linkbordercolor={0 0 1}
}

\lstdefinestyle{Python}{
	language        = Python,
	frame           = lines, 
	basicstyle      = \footnotesize,
	keywordstyle    = \color{blue},
	stringstyle     = \color{green},
	commentstyle    = \color{red}\ttfamily
}

\setlength{\parindent}{0.0in}
\setlength{\parskip}{0.05in}

\newcommand\course{\textbf{EE 456}}   
\newcommand\name{Aishwarye Omer}     

\pagestyle{fancyplain}
\headheight 35pt
\lhead{\name\\\course{}}
\chead{\textbf{\Large Homework - 12}}
\rhead{\today}
\lfoot{}
\cfoot{}
\rfoot{\small\thepage}
\headsep 1.5em

\newlength\myindent
\setlength\myindent{2em}
\newcommand\bindent{%
	\begingroup
	\setlength{\itemindent}{\myindent}
	\addtolength{\algorithmicindent}{\myindent}
}
\newcommand\eindent{\endgroup}

\newenvironment{solution}[1][Solution]{\begin{trivlist}
		\item[\hskip \labelsep {\bfseries #1}]}{\end{trivlist}}



\definecolor{lightgray}{gray}{0.5}
\setlength{\parindent}{0pt}

\begin{document}
\textbf{Question:} For what value of $\epsilon$, the network would converge in 5 cycles or less.
\BlankLine
\textbf{Solution:} 
\begin{itemize}
	\item I chose $\epsilon$ represented in code by $e$ as $e \ = \ -0.2$ because there are 5 inputs i.e. $m \ = \ 5$. The general rule to select $e$ is that $0 \leq e \leq 1/m$. 
	
	\item $1/m \ = \ 1/5 = 0.2$
\end{itemize}
Following is the code  and result of the network:
\BlankLine
\BlankLine
    
    \begin{verbatim}
e = -0.2;
w_ii = 1.2;
Y1 = 0.1;
Y2 = 0.3;
Y3 = 0.7;
Y4 = 0.5;
Y5 = 0.2;

for i = 1:5
    Y1new = round(w_ii*Y1 + e*(Y2+Y3+Y4+Y5),2);
    Y2new = round(w_ii*Y2 + e*(Y1+Y3+Y4+Y5),2);
    Y3new = round(w_ii*Y3 + e*(Y1+Y2+Y4+Y5),2) ;
    Y4new = round(w_ii*Y4 + e*(Y1+Y2+Y3+Y5),2);
    Y5new = round(w_ii*Y5 + e*(Y1+Y2+Y3+Y4),2);
    if Y1new<=0
        Y1=0;
        fprintf('Y1 = %f\t\n', Y1);
    else
        Y1 = Y1new;
        fprintf('Y1 = %f\t\n', Y1);
    end
    if Y2new<=0
        Y2 = 0;
        fprintf('Y2 = %f\t\n', Y2);
    else
        Y2 = Y2new;
        fprintf('Y2 = %f\t\n', Y2);
    end

    if Y3new<=0
        Y3 = 0;
        fprintf('Y3 = %f\t\n', Y3);
    else
        Y3 = Y3new;
        fprintf('Y3 = %f\t\n', Y3);
    end

    if Y4new<=0
        Y4 = 0;
        fprintf('Y4 = %f\t\n', Y4);
    else
        Y4 = Y4new;
        fprintf('Y4 = %f\t\n', Y4);
    end

    if Y5new<=0
        Y5 = 0;
        fprintf('Y5 = %f\t\n', Y5);
    else
        Y5 = Y5new;
        fprintf('Y5 = %f\t\n', Y5);
    end
   fprintf('\n');

end
\end{verbatim}

        \color{lightgray} \begin{verbatim}
Y1 = 0.000000	
Y2 = 0.060000	
Y3 = 0.620000	
Y4 = 0.340000	
Y5 = 0.000000	

Y1 = 0.000000	
Y2 = 0.000000	
Y3 = 0.660000	
Y4 = 0.270000	
Y5 = 0.000000	

Y1 = 0.000000	
Y2 = 0.000000	
Y3 = 0.740000	
Y4 = 0.190000	
Y5 = 0.000000	

Y1 = 0.000000	
Y2 = 0.000000	
Y3 = 0.850000	
Y4 = 0.080000	
Y5 = 0.000000	

Y1 = 0.000000	
Y2 = 0.000000	
Y3 = 1.000000	
Y4 = 0.000000	
Y5 = 0.000000	

\end{verbatim}   

\color{black}\text{After 5 cycles, everything becomes zero except $Y3$}

\end{document}

